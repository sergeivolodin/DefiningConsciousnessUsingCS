\documentclass[a4,left=1cm]{article}
\usepackage{hyperref}
%opening
\title{Philosophy of Life Sciences. Essay}
\author{Sergei Volodin, EPFL MSc student}

\begin{document}

\maketitle

\begin{abstract}
This essay claims that Free Will exists by linking human behavior with Reinforcement Learning framework and proving existence of free will inside that framework
\end{abstract}

\section{Introduction}
{\bf Claim:} free will can exist since human behavior fits Reinforcement Learning model, which supports free will in terms of random input for an algorithm

The steps are the following:
\begin{enumerate}
	\item What is Free Will?
	\item What is Reinforcement Learning?
	\item Why does human behavior fit Reinforcement Learning paradigm?
	\item If so, how can free will exist if actions of humans can be predicted (randomness) $\rightarrow$ incompatibility of free will and determinism (free will requires randomness)
	\item Why does Reinforcement Learning admit Free Will?
	\item Conclusion (free will)
\end{enumerate}

Free will is a philosophical concept stating that humans can take decisions not considering data as previous experience or impediments.


Reinforcement learning is a subfield of Machine Learning and Artificial Intelligence aimed at replicating behavior of an agent in an environment \cite{sutton}

Suppose that we live in a world where AGI exists, or, at least, human actions can be explained using RL. Now it is able to explain actions of simple mammals such as mice \cite{oist}.

The problem with free will would be that RL might seem to be able to explain human behaviour and actions humans take, if not in each case, then on average.

In this case, even if Free Will does not seem to exist (one of the explanations would be that algorithms need random input, each bit of them can be interpreted as free will decision -- but this idea needs further elaboration since not all algorithms require such data)

Isn't randomness implies negation of free will? If non-determinism is true, than we certainly prove that actions of humans can be random => incompatibility

RL agents and consciousness \cite{rlmorality1}, \cite{rlmorality2}

Thus, we will show that free will can exist if we prove that RL indeed fit human behavior.

\begin{thebibliography}{20}
\bibitem{sutton} Richard S. Sutton and Andrew G. Barto. 1998. Introduction to Reinforcement Learning (1st ed.). MIT Press, Cambridge, MA, USA.
\bibitem{oist} \url{https://groups.oist.jp/ncu} (late 2010s)
\bibitem{freewillrl} Kenneth T. Kishida. A computational approach to “free will” constrained by the games we play
\bibitem{rlmorality1} Brian Tomasik. Do Artificial Reinforcement-Learning Agents Matter Morally?
\bibitem{rlmorality2} \url{http://reducing-suffering.org/ethical-issues-artificial-reinforcement-learning/}
\bibitem{1} \url{http://oxfordindex.oup.com/view/10.1093/acprof:oso/9780198713241.003.0003}
\bibitem{2} Negation of Free will: \url{//www.informationphilosopher.com/freedom/standard_argument.html}
\bibitem{3} \url{https://plato.stanford.edu/entries/freewill/}
\bibitem{4} \url{http://serendip.brynmawr.edu/sci_cult/evolit/s05/web2/lpaterek.html}
\bibitem{5} \url{https://www.researchgate.net/publication/278468286_Creating_Free_Will_in_Artificial_Intelligence}
\bibitem{6} \url{https://philosophy.stackexchange.com/questions/36639/in-what-type-of-world-is-free-will-possible-if-at-all}
\bibitem{7} \url{https://becominghuman.ai/can-a-i-ever-have-free-will-c18b4f489b45}
\bibitem{8} \url{https://philpapers.org/browse/philosophy-of-artificial-intelligence}
\bibitem{9} \url{https://en.wikipedia.org/wiki/Philosophy_of_artificial_intelligence}
\bibitem{a} \url{https://www.theguardian.com/science/2012/oct/03/philosophy-artificial-intelligence}
\bibitem{b} \url{http://jmc.stanford.edu/articles/aiphil2.html}
\bibitem{c} \url{https://plato.stanford.edu/entries/logic-ai/}
\bibitem{d} \url{https://arxiv.org/abs/1605.06048}
\bibitem{e} \url{https://www.youtube.com/watch?v=39EdqUbj92U}
\bibitem{f} \url{https://www.sciencedirect.com/science/article/pii/B9780934613033500337}
\bibitem{g} \url{https://link.springer.com/article/10.1007/s00221-013-3402-y}
\bibitem{h} \url{https://plato.stanford.edu/entries/incompatibilism-theories/}
\bibitem{i} \url{https://github.com/lucasdupin/game_theory_of_life}
\bibitem{j} \url{https://plato.stanford.edu/entries/freewill/}
\bibitem{k} \url{http://andrewmbailey.com/pvi/van_Inwagen_on_Free_Will.pdf}
\bibitem{l} \url{https://www.nature.com/news/2011/110831/full/477023a.html}
\bibitem{m} \url{http://www.sciencedirect.com/science/article/pii/0004370271900117}
\bibitem{n} \url{http://sro.sussex.ac.uk/17112/}
\bibitem{o} \url{https://link.springer.com/book/10.1007\%2F978-3-642-31674-6}
\bibitem{p} \url{https://pdfs.semanticscholar.org/d9f9/91f3a9bc95c9b58dcb8d84aeedb40122ce37.pdf}
\bibitem{q} \url{http://www.sophia.de/pdf/2012_M&M_PT-AI_Introduction.pdf}
\bibitem{r} \url{http://www-formal.stanford.edu/jmc/aiphil.pdf}
\bibitem{s} \url{https://ai.stanford.edu/~nilsson/QAI/qai.pdf}
\bibitem{t} Free will and randomness \url{https://philosophy.stackexchange.com/questions/1012/what-is-the-difference-between-free-will-and-randomness-and-or-non-determinism} \url{https://www.quora.com/What-is-the-difference-between-free-will-and-randomness} \url{https://www.scientificamerican.com/article/quantum-physics-free-will/}
\end{thebibliography}
\end{document}
